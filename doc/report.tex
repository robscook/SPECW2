\documentclass[a4paper,11pt]{report}
\usepackage[T1]{fontenc}
\usepackage[utf8]{inputenc}
\usepackage{lmodern}

\title{SPE: CW2: Performance Modelling}
\author{Efe Ovadje (K1763735), Robert Cook (K1763733), Emmanuel Buraimo (K1630440)}
\date{2020}

\begin{document}

\maketitle
%\tableofcontents

\chapter*{Introduction}
We have been commissioned
by a financial services company,
to assess the feasibility of a system migration of a
legacy mainframe to the more recent Sun Solaris and
J2EE platform, while providing the expertise and guidance necessary.

A new application titled, 'Payments and Clearing' is expected to
utilize this new system.
It receives payments via \textit{http} from three separate business channels.
The application is comprised of four major components.
These are \textit{Inbound}, \textit{Message processing},
\textit{Processing and exception management} and the \textit{Oracle Server}.

The primary KPS identified during the performance and
scalability assessment was 'Process Peak Hour Current Processing Day Payments'.
There are two NFR's for this application,
'File Inter Arrival Time secs' (NFR1) and
'File 95 th Response Time secs' (NFR2).

\chapter*{Modelling}

\section*{Inbound}

\section*{Message processing}

\section*{Processing and exception management (PEM)}

\section*{Oracle server}
The Oracle server provided read and write functions that could access and manipulate application data, this was through the implementation of five procedures. These procedures are then accessed through the different application components, namely; Inbound Processing, Message Processing and PEM.\linebreak

When we started the project, the basic architecture of the Oracle server had been implemented. Our task was to build upon this architecture and model how the payment system used the Oracle server when a file was submitted into the system. Within the Oracle server Interface, we created the five procedures that would provide the necessary functionality. Connected to the Oracle Server Interface was a module named "Basic Component". The purpose of this component was to be able to assign specifically the action and the resource demand that each procedure would have. Through the use of SEFF diagrams, we could visually see how the different procedures would act and their resource consumption.

Based on the project brief that we were given, we modelled each procedure as follows; Procedure 1 had a resource demand of 500 CPU cycles per read business rules action from Oracle table 1, and an HDD of 2KB; Procedure 2 had a resource demand of 500 CPU cycles per write payment data action to Oracle table 2, and an HDD of 1KB; Procedure 3 had a resource demand of 2000 CPU cycles per backup of submission file action, and an HDD of 10MB; Procedure 4 had a resource demand of 50 CPU cycles per write exception action, and an HDD of 1KB; lastly Procedure 5 had a resource demand of 50 CPU cycles per write exception referral action, and an HDD of 1KB. 

\chapter*{Simulation}

\section*{SPESIM1}

\section*{SPESIM2}

\chapter*{JMS modelling}

\end{document}
